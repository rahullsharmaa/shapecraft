\documentclass{beamer}
\usetheme{Madrid}
\usecolortheme{default}
\title{Complex Numbers}
\author{Rahul Sharma}
\institute{Shyam Lal College}
\date{\today}

\begin{document}

\begin{frame}
\titlepage
\end{frame}

\begin{frame}{Introduction}
    \begin{itemize}
        \item Definition of complex numbers
        \item Real and imaginary parts of complex numbers
        \item Properties of complex numbers
    \end{itemize}
\end{frame}

\begin{frame}{Definition of Complex Numbers}
    A complex number is a number of the form $a + bi$, where $a$ and $b$ are real numbers and $i$ is the imaginary unit, which satisfies $i^2 = -1$.

    and i = $\sqrt{-1}$,
        $i^2 = -1$,
        $i^3 = -i$,
        $i^4 = 1$.
\end{frame}

\begin{frame}{Real and Imaginary Parts of Complex Numbers}
    The real part of a complex number $a + bi$ is $a$, and the imaginary part is $b$.
\end{frame}

\begin{frame}{Properties of Complex Numbers}
    \begin{itemize}
        \item Addition and subtraction of complex numbers
        \item Multiplication and division of complex numbers
        \item Modulus and argument of complex numbers
        \item Complex conjugate of a complex number
    \end{itemize}
\end{frame}

\begin{frame}{Addition and Subtraction of Complex Numbers}
    \begin{block}{Example 1}
        To add two complex numbers:
        \begin{align*}
            (3 + 2i) + (4 - i) &= (3 + 4) + (2 - 1)i \\
            &= 7 + i
        \end{align*}
    \end{block}
    \begin{block}{Example 2}
        To subtract two complex numbers:
        \begin{align*}
            (5 - 3i) - (2 + 4i) &= (5 - 2) - (3 + 4)i \\
            &= 3 - 7i
        \end{align*}
    \end{block}
\end{frame}

\begin{frame}{Multiplication and Division of Complex Numbers}
    \begin{block}{Example 3}
        To multiply two complex numbers:
        \begin{align*}
            (2 + i)(3 - 2i) &= 6 - 4i + 3i - 2i^2 \\
            &= 6 - i + 2 \\
            &= 8 - i
        \end{align*}
    \end{block}
    \begin{block}{Example 4}
        To divide two complex numbers:
        \begin{align*}
            \frac{1 + 2i}{4 - i} &= \frac{(1 + 2i)(4 + i)}{(4 - i)(4 + i)} \\
            &= \frac{4 + 8i + i + 2i^2}{16 - i^2} \\
            &= \frac{2 + 9i}{17}
        \end{align*}
    \end{block}
\end{frame}

\begin{frame}{Modulus and Argument of Complex Numbers}
    \begin{block}{Example 5}
        The modulus of a complex number $a + bi$ is the distance from the origin to the point $(a, b)$:
        \[
            |a + bi| = \sqrt{a^2 + b^2}
        \]
    \end{block}
    \begin{block}{Example 6}
        The argument of a complex number $a + bi$ is the angle between the positive real axis and the line to $(a, b)$:
        \[
            \arg(a + bi) = \tan^{-1}\left(\frac{b}{a}\right)
        \]
    \end{block}
\end{frame}

\begin{frame}{Complex Conjugate of a Complex Number}
    \begin{block}{Example 7}
        The complex conjugate of $a + bi$ is $a - bi$:
        \[
            \overline{a + bi} = a - bi
        \]
    \end{block}
    \begin{block}{Example 8}
        For any complex number $z$, the product of $z$ and its conjugate is real:
        \[
            z \cdot \overline{z} = (a + bi)(a - bi) = a^2 + b^2
        \]
    \end{block}
\end{frame}

\end{document}
